\renewcommand{\Color}[1]{}

\newenvironment{etabular}
  {\edtable{tabular}}
  {\endedtable}

\newcommand{\semanticstwo}[2]
{
\Color{green}
\begin{etabular}{l|l}
\hspace*{0.1cm} & #1 \\
& #2\\
\end{etabular}
\vspace*{0.2cm}
\Color{black}
}

\newcommand{\semanticsthree}[3]
{
\Color{green}
\begin{etabular}{l|l}
\hspace*{0.1cm} & #1 \\
& #2\\
& #3\\
\end{etabular}
\vspace*{0.2cm}
\Color{black}
}

\newcommand{\semanticsfour}[4]
{
\Color{green}
\begin{etabular}{l|l}
\hspace*{0.1cm} & #1 \\
& #2\\
& #3\\
& #4\\
\end{etabular}
\vspace*{0.2cm}
\Color{black}
}

\chapter{Overview}
\label{chap:overview}

Modern computing systems contain a variety of memory types, each
closely associated with a distinct type of computing hardware. For
example, compute accelerators such as GPUs typically feature their
own memory that is distinct from the memory attached to the host
processor. Additionally, GPUs from different vendors also differ
in their memory types. The differences in memory types influence
feature availability and performance behavior of an application
running on such modern systems. Hence, \mpi/ libraries need to be
aware of and support additional memory types. For a given type of
memory, \mpi/ libraries need to know the associated memory allocator and
the limitations on memory access. The different memory kinds capture
the differentiating information needed by \mpi/ libraries for different
memory types.

This \mpi/ side document defines the memory allocation kinds and their
associated restrictors that users can use to query the support for
different memory kinds provided by the \mpi/ library. These definitions
supplement those found in section 11.4.3 of the \mpi/ standard, which
also explains their usage model.

\chapter{Definitions}
\label{chap:definitions}

This section contains definitions of memory allocation kinds and
their restrictors.

\section{Kind: cuda}

The \infokey{cuda} memory kind refers to the memory allocated by the
CUDA runtime system~\cite{cudaref}. \exref{example:alloc-kind-spm-cuda}
showcases usage of some of the memory kinds defined in this section. 

\subsubsection{Restrictors}

\begin{itemize}

\item \infokey{host}: Support for memory allocations on the host system
    that are page-locked for direct access from the CUDA device (\eg,
        memory allocations from the \function{cudaHostAlloc()} function).
        These memory allocations are attributed with \function{cudaMemoryTypeHost}.

\item \infokey{device}: Support for memory allocated on a CUDA device
    (\eg, memory allocations from the \function{cudaMalloc()} function).
        These memory allocations are attributed with \function{cudaMemoryTypeDevice}.

\item \infokey{managed}: Support for memory that is managed by CUDA’s
    Unified Memory system (\eg, memory allocations from the
        \function{cudaMallocManaged()} function).
        These memory allocations are attributed with \function{cudaMemoryTypeManaged}.

\end{itemize}

\section{Kind: rocm}

The \infokey{rocm} memory kind refers to the memory allocated by the ROCm
runtime system~\cite{rocmref}.

\subsubsection{Restrictors}

\begin{itemize}

\item \infokey{host}: Support for memory allocated on the host system that
    is page-locked for direct access from the ROCm device (\eg, memory
        allocations from the \function{hipHostMalloc()} function).
        These memory allocations are attributed with \function{hipMemoryTypeHost}.

\item \infokey{device}: Support for memory allocated on the ROCm device
    (\eg, memory allocations from the \function{hipMalloc()} function).
        These memory allocations are attributed with \function{hipMemoryTypeDevice}.

\item \infokey{managed}: Support for memory that is managed automatically
    by the ROCm runtime (\eg, memory allocations from the
        \function{hipMallocManaged()} function).
        These memory allocations are attributed with \function{hipMemoryTypeManaged}.

\end{itemize}

\section{Kind: level\_zero}

The \infokey{level\_zero} memory kind refers to the memory allocated by the
Level Zero runtime system~\cite{zeref}.

\subsubsection{Restrictors}

\begin{itemize}

\item \infokey{host}: Support for memory allocated on the host that is
    accessible by Level Zero devices (\eg, memory allocations from the
        \function{zeMemAllocHost()} function).
        These memory allocations are attributed with \function{ZE\_MEMORY\_TYPE\_HOST}.

\item \infokey{device}: Support for memory allocated on a Level Zero device
    (\eg, memory allocations from the \function{zeMemAllocDevice()} function).
        These memory allocations are attributed with \function{ZE\_MEMORY\_TYPE\_DEVICE}.

\item \infokey{shared}: Support for memory allocated that will be shared
    between the host and one or more Level Zero devices (\eg,
        memory allocations from the \function{zeMemAllocShared()} function).
        These memory allocations are attributed with \function{ZE\_MEMORY\_TYPE\_SHARED}.

\end{itemize}

\chapter{Examples}
\label{chap:examples}

This section includes examples demonstrating the usage of
memory kinds defined in \chapref{chap:definitions}.

\begin{example}
\label{example:alloc-kind-spm-cuda}
This CUDA example demonstrates the usage of the different
kinds to perform communication in a manner that is supported
by the underlying \mpi/ library.
%%HEADER
%%LANG: C
%%ENDHEADER
\begin{lstlisting}[language={[MPI]C}]
#include <stdio.h>
#include <stdlib.h>
#include <string.h>
#include <assert.h>
#include <mpi.h>
#include <cuda_runtime.h>

int main(int argc, char *argv[])
{
    int cuda_device_aware = 0;
    int cuda_managed_aware = 0;
    int len = 0, flag = 0;
    int *managed_buf = NULL;
    int *device_buf = NULL, *system_buf = NULL;
    int nranks = 0;
    MPI_Info info;
    MPI_Session session;
    MPI_Group wgroup;
    MPI_Comm system_comm;
    MPI_Comm cuda_managed_comm = MPI_COMM_NULL;
    MPI_Comm cuda_device_comm = MPI_COMM_NULL;

    MPI_Info_create(&info);
    MPI_Info_set(info, "mpi_memory_alloc_kinds",
                       "system,cuda:device,cuda:managed");
    MPI_Session_init(info, MPI_ERRORS_ARE_FATAL, &session);
    MPI_Info_free(&info);

    MPI_Session_get_info(session, &info);
    MPI_Info_get_string(info, "mpi_memory_alloc_kinds",
                        &len, NULL, &flag);

    if (flag) {
        char *val, *valptr, *kind;

        val = valptr = (char *) malloc(len);
        if (NULL == val) return 1;

        MPI_Info_get_string(info, "mpi_memory_alloc_kinds",
                            &len, val, &flag);

        while ((kind = strsep(&val, ",")) != NULL) {
            if (strcasecmp(kind, "cuda:managed") == 0) {
                cuda_managed_aware = 1;
            }
            else if (strcasecmp(kind, "cuda:device") == 0) {
                cuda_device_aware = 1;
            }
        }
        free(valptr);
    }

    MPI_Info_free(&info);

    MPI_Group_from_session_pset(session, "mpi://WORLD" , &wgroup);

    // Create a communicator for operations on system memory
    MPI_Info_create(&info);
    MPI_Info_set(info, "mpi_assert_memory_alloc_kinds", "system");
    MPI_Comm_create_from_group(wgroup,
            "org.mpi-side-doc.mem-kind.example.system",
            info, MPI_ERRORS_ABORT, &system_comm);
    MPI_Info_free(&info);

    MPI_Comm_size(system_comm, &nranks);

    /** Check for CUDA awareness **/
    // Check if all processes have CUDA managed support
    MPI_Allreduce(MPI_IN_PLACE, &cuda_managed_aware, 1, MPI_INT,
                  MPI_LAND, system_comm);

    if (cuda_managed_aware) {
        // Create a communicator for operations that use
        // CUDA managed buffers.
        MPI_Info_create(&info);
        MPI_Info_set(info, "mpi_assert_memory_alloc_kinds",
                     "cuda:managed");
        MPI_Comm_create_from_group(wgroup,
            "org.mpi-side-doc.mem-kind.example.cuda.managed",
            info, MPI_ERRORS_ABORT, &cuda_managed_comm);
        MPI_Info_free(&info);
    }
    else {
        // Check if all processes have CUDA device support
        MPI_Allreduce(MPI_IN_PLACE, &cuda_device_aware, 1, MPI_INT,
                      MPI_LAND, system_comm);
        if (cuda_device_aware) {
            // Create a communicator for operations that use
            // CUDA device buffers.
            MPI_Info_create(&info);
            MPI_Info_set(info, "mpi_assert_memory_alloc_kinds",
                         "cuda:device");
            MPI_Comm_create_from_group(wgroup,
                "org.mpi-side-doc.mem-kind.example.cuda.device",
                info, MPI_ERRORS_ABORT, &cuda_device_comm);
            MPI_Info_free(&info);
        }
        else {
            printf("Warning: cuda alloc kind not supported\n");
        }
    }

    MPI_Group_free(&wgroup);

    /** Execute according to level of CUDA awareness **/
    if (cuda_managed_aware) {
        // Allocate managed buffer and initialize it
        cudaMallocManaged((void**)&managed_buf, sizeof(int));
        *managed_buf = 1;

        // Perform communication using cuda_managed_comm
        // if it's available.
        MPI_Allreduce(MPI_IN_PLACE, managed_buf, 1, MPI_INT,
                      MPI_SUM, cuda_managed_comm);

        assert((*managed_buf) == nranks);

        cudaFree(managed_buf);
    }
    else {
        // Allocate system buffer and initialize it
        system_buf = (int*)malloc(sizeof(int));
        *system_buf = 1;

        // Allocate CUDA device buffer and initialize it
        cudaMalloc((void**)&device_buf, sizeof(int));
        cudaMemcpyAsync(device_buf, system_buf, sizeof(int),
                        cudaMemcpyHostToDevice);

        cudaStreamSynchronize(0);
        if (cuda_device_aware) {
            // Perform communication using cuda_device_comm
            // if it's available.
            MPI_Allreduce(MPI_IN_PLACE, device_buf, 1, MPI_INT,
                          MPI_SUM, cuda_device_comm);

            assert((*device_buf) == nranks);
        }
        else {
            // Otherwise, copy data to a system buffer,
            // use system_comm, and copy data back to device buffer
            cudaMemcpyAsync(system_buf, device_buf, sizeof(int),
                            cudaMemcpyDeviceToHost);

            cudaStreamSynchronize(cudaStreamDefault);
            MPI_Allreduce(MPI_IN_PLACE, system_buf, 1, MPI_INT,
                          MPI_SUM, system_comm);
            cudaMemcpyAsync(device_buf, system_buf, sizeof(int),
                            cudaMemcpyHostToDevice);

            cudaStreamSynchronize(cudaStreamDefault);
            assert((*system_buf) == nranks);
        }

        cudaFree(device_buf);
        free(system_buf);
    }

    if (cuda_managed_comm != MPI_COMM_NULL)
        MPI_Comm_disconnect(&cuda_managed_comm);
    if (cuda_device_comm != MPI_COMM_NULL)
        MPI_Comm_disconnect(&cuda_device_comm);
    MPI_Comm_disconnect(&system_comm);

    MPI_Session_finalize(&session);

    return 0;
}
\end{lstlisting}
\end{example}

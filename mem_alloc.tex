%\chapter{The MPIR Process Acquisition Interface}

\renewcommand{\Color}[1]{}

\newenvironment{etabular}
  {\edtable{tabular}}
  {\endedtable}

\newcommand{\semanticstwo}[2]
{
\Color{green}
\begin{etabular}{l|l}
\hspace*{0.1cm} & #1 \\
& #2\\
\end{etabular}
\vspace*{0.2cm}
\Color{black}
}

\newcommand{\semanticsthree}[3]
{
\Color{green}
\begin{etabular}{l|l}
\hspace*{0.1cm} & #1 \\
& #2\\
& #3\\
\end{etabular}
\vspace*{0.2cm}
\Color{black}
}

\newcommand{\semanticsfour}[4]
{
\Color{green}
\begin{etabular}{l|l}
\hspace*{0.1cm} & #1 \\
& #2\\
& #3\\
& #4\\
\end{etabular}
\vspace*{0.2cm}
\Color{black}
}

\chapter{Overview}

Modern computing systems contain a variety of memory types, each
closely associated with a distinct type of computing hardware. For
example, compute accelerators such as GPUs typically feature their
own memory that is distinct from the memory attached to the host
processor. Additionally, GPUs from different vendors also differ
in their memory types. The differences in memory types influence
feature availability and performance behavior of an application
running on such modern systems. Hence, MPI libraries need to be
aware of and support additional memory types. For a given type of
memory, MPI libraries need to know the associated memory allocator and
the limitations on memory access. The different memory kinds capture
the differentiating information needed by MPI libraries for different
memory types. 

This MPI side document defines the memory allocation kinds and their
associated restrictors that users can use to query the support for
different memory kinds provided by the MPI library. These definitions
supplement those found in section 11.4.3 of the MPI standard, which
also explains their usage model. 

\chapter{Definitions}

This section contains definitions of memory allocation kinds and
their restrictors.

\section{Kind: cuda}

The \infokey{cuda} memory kind refers to the memory allocated by the
CUDA runtime system~\cite{cudaref}.

\subsubsection{Restrictors}

\begin{itemize}

\item \infokey{host}: Support for memory allocations on the host system
    that are page-locked for direct access from the CUDA device (\eg,
        memory allocations from the \function{cudaHostAlloc()} function).

\item \infokey{device}: Support for memory allocated on a CUDA device
    (\eg, memory allocations from the \function{cudaMalloc()} function).

\item \infokey{managed}: Support for memory that is managed by CUDA’s
    Unified Memory system (\eg, memory allocations from the
        \function{cudaMallocManaged()} function).

\end{itemize}

\section{Kind: rocm}

The \infokey{rocm} memory kind refers to the memory allocated by the ROCm
runtime system~\cite{rocmref}.

\subsubsection{Restrictors}

\begin{itemize}

\item \infokey{host}: Support for memory allocated on the host system that
    is page-locked for direct access from the ROCm device (\eg, memory
        allocations from the \function{hipHostMalloc()} function).

\item \infokey{device}: Support for memory allocated on the ROCm device
    (\eg, memory allocations from the \function{hipMalloc()} function).

\item \infokey{managed}: Support for memory that is managed automatically
    by the ROCm runtime (\eg, memory allocations from the
        \function{hipMallocManaged()} function).

\end{itemize}

\section{Kind: levelzero}

The \infokey{levelzero} memory kind refers to the memory allocated by the
Level Zero runtime system~\cite{zeref}.

\subsubsection{Restrictors}

\begin{itemize}

\item \infokey{host}: Support for memory allocated on the host that is
    accessible by Level Zero devices (\eg, memory allocations from the
        \function{zeMemAllocHost()} function).

\item \infokey{device}: Support for memory allocated on a Level Zero device
    (\eg, memory allocations from the \function{zeMemAllocDevice()} function). 

\item \infokey{shared}: Support for memory allocated that will be shared
    between the host and one or more Level Zero devices (\eg,
        memory allocations from the \function{zeMemAllocShared()} function).

\end{itemize}

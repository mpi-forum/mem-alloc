{\Huge\noindent\textbf{Acknowledgments}}%%ALLOWLATEX%
\phantomsection
\addcontentsline{toc}{chapter}{Acknowledgments}

\vspace{0.5in}

%This document is the product of a number of
%  distinct efforts in four distinct phases: one for each of \MPII/,
%  \MPIII/, \MPIIII/, and \MPIIV/.  This section describes these in historical
%  order, starting with \MPII/.  Some efforts, particularly parts of
%  \MPIII/, had distinct groups of individuals associated with them,
%  and these efforts are detailed separately.

%\vspace{0.5in}

This document represents the work of many people who have served on
the \MPIHACC/.  The meetings have been attended by dozens of people
from many parts of the world. It is the hard and dedicated work of
this group that has led to the \MEMALLOCDOC/ document. The technical
development was carried out by subgroups, whose work was reviewed by
the full committee.

Those who served as primary coordinators in Version 1.0 are:

\begin{itemize}
\item James Dinan, Rohit Zambre, NVIDIA CUDA
\item Pedram Alizadeh, Edgar Gabriel, Michael Klemm, AMD ROCm
\item Maria Garzaran, Dan Holmes, Intel Level Zero
\end{itemize}

The following list includes some of the active participants in
the \MPIHACC/.

\begin{center}
\begin{tabular}{llll}
Pedram Alizadeh &
James Dinan &
Dmitry Durnov &
Edgar Gabriel \\
Maria Garzaran &
Yanfei Guo &
Khaled Hamidouche &
Dan Holmes \\
Michael Klemm &
Michael Knobloch &
Guillaume Mercier &
Christoph Niethammer \\
Howard Pritchard &
Ken Raffenetti &
Brian Smith &
Joseph Schuchart \\
Amirhossein Sojoodi &
Yiltan Temuçin &
Rohit Zambre &
Hui Zhou
\end{tabular}
\end{center}

The following institutions supported the Version 1.0 effort
of the \MEMALLOCDOC/ through time and travel support for the people
listed above.

\medskip

\begin{obeylines}\leftskip=\parindent\parindent=0pt %ALLOWLATEX%
Advanced Micro Devices, Inc.
Amazon.com, Inc
Argonne National Laboratory
Cornelis Networks
Forschungszentrum Jülich
HLRS, University of Stuttgart
Institut National de Recherche en Informatique et Automatique (Inria)
Intel Corporation
Los Alamos National Laboratory
NVIDIA Corporation
Queen's University
University of Tennessee, Knoxville
\end{obeylines}

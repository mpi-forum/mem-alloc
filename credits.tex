{\Huge\noindent\textbf{Acknowledgments}}%%ALLOWLATEX%
\phantomsection
\addcontentsline{toc}{chapter}{Acknowledgments}

\vspace{0.5in}

%This document is the product of a number of
%  distinct efforts in four distinct phases: one for each of \MPII/,
%  \MPIII/, \MPIIII/, and \MPIIV/.  This section describes these in historical
%  order, starting with \MPII/.  Some efforts, particularly parts of
%  \MPIII/, had distinct groups of individuals associated with them,
%  and these efforts are detailed separately.

%\vspace{0.5in}

This document represents the work of many people who have served on
the \MPIHACC/.  The meetings have been attended by dozens of people
from many parts of the world. It is the hard and dedicated work of
this group that has led to the \MEMALLOCDOC/ document. The technical
development was carried out by subgroups, whose work was reviewed by
the full committee.

Those who served as primary coordinators in Version 1.0 are:

\begin{itemize}
\item Rohit Zambre, James Dinan, Nvidia CUDA
\item Edgar Gabriel, Pedram Alizadeh, Michael Klemm, AMD ROCm
\item Maria Garzaran, Daniel Holmes, Intel Level Zero
\end{itemize}

The following list includes some of the active participants in
the \MPIHACC/.

\begin{center}
\begin{tabular}{llll}
Kenneth Raffenetti     &
Joseph Schuart	&
Hui Zhou &
Khaled Hamidouche
\end{tabular}
\end{center}

\begin{comment}
The \MPI/ Forum also acknowledges and appreciates the valuable input
from people via e-mail and in person.

% People who provided significant input during the public comment
%The \MPI/ Forum also thanks those that provided feedback during the
%public comment period.

The following institutions supported the \MPIIVDOTI/ effort through
time and travel support for the people listed above.

\medskip

\begin{obeylines}\leftskip=\parindent\parindent=0pt %ALLOWLATEX%
Advanced Micro Devices, Inc.
Amazon.com, Inc
Argonne National Laboratory
Atos
CEA
Cisco Systems Inc.
Collis-Holmes Innovations Limited
Cornelis Networks
EPCC, The University of Edinburgh
Forschungszentrum J\"ulich
Fujitsu
HLRS, University of Stuttgart
Hewlett Packard Enterprise
Institut National de Recherche en Informatique et Automatique (Inria)
Intel Corporation
International Business Machines
Kichakato Kizito
Lawrence Berkeley National Laboratory
Lawrence Livermore National Laboratory
Leibniz Supercomputing Centre
Los Alamos National Laboratory
Meta Platforms Inc.
NVIDIA Corporation
Oak Ridge National Laboratory
ParaTools SAS
Queen's University
RWTH Aachen University
Sandia National Laboratory
Technical University of Munich
Tennessee Technological University
Texas Advanced Computing Center
The Ohio State University
University of Alabama
University of Alabama at Birmingham
University of Basel
University of Illinois Urbana-Champaign
University of New Mexico
University of Tennessee, Chattanooga
University of Tennessee, Knoxville
University of Tokyo
Université Grenoble Alpes
VSC Research Center, TU Wien
ZIH, TU Dresden
\end{obeylines}
\end{comment}

 
\newpage
\phantomsection
\addcontentsline{toc}{chapter}{Examples Index}
\makeatletter
\renewcommand{\@index}{Examples Index}
\renewcommand{\@introtext}{This index lists code examples throughout the text.  Some examples are referred to by content; others are listed by the major \MPI/ function that they are demonstrating.  \MPI/ functions listed in all capital letter are Fortran examples; \MPI/ functions listed in mixed case are C/C++ examples.}
\makeatother
 
\newpage
\phantomsection
\addcontentsline{toc}{chapter}{MPI Constant and Predefined Handle Index}
\makeatletter
\renewcommand{\@index}{MPI Constant and Predefined Handle Index}
\renewcommand{\@introtext}{This index lists predefined \MPI/ constants and handles.}
\makeatother
 
\newpage
\phantomsection
\addcontentsline{toc}{chapter}{MPI Declarations Index}
\makeatletter
\renewcommand{\@index}{MPI Declarations Index}
\renewcommand{\@introtext}{This index refers to declarations needed in C/C++, such as address kind integers, handles, etc. The underlined page numbers is the ``main'' reference (sometimes there are more than one when key concepts are discussed in multiple areas).}
\makeatother
 
\newpage
\phantomsection
\addcontentsline{toc}{chapter}{MPI Callback Function Prototype Index}
\makeatletter
\renewcommand{\@index}{MPI Callback Function Prototype Index}
\renewcommand{\@introtext}{This index lists the C typedef names for callback routines, such as those used with attribute caching or user-defined reduction operations.  C++ names for these typedefs and Fortran example prototypes are given near the text of the C name.}
\makeatother
 
\newpage
\phantomsection
\addcontentsline{toc}{chapter}{MPI Function Index}
\makeatletter
\renewcommand{\@index}{MPI Function Index}
\renewcommand{\@introtext}{The underlined page numbers refer to the function definitions.}
\makeatother
